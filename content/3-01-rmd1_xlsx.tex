% Options for packages loaded elsewhere
\PassOptionsToPackage{unicode}{hyperref}
\PassOptionsToPackage{hyphens}{url}
\PassOptionsToPackage{dvipsnames,svgnames,x11names}{xcolor}
%
\documentclass[
  letterpaper,
  DIV=11,
  numbers=noendperiod]{scrartcl}

\usepackage{amsmath,amssymb}
\usepackage{lmodern}
\usepackage{iftex}
\ifPDFTeX
  \usepackage[T1]{fontenc}
  \usepackage[utf8]{inputenc}
  \usepackage{textcomp} % provide euro and other symbols
\else % if luatex or xetex
  \usepackage{unicode-math}
  \defaultfontfeatures{Scale=MatchLowercase}
  \defaultfontfeatures[\rmfamily]{Ligatures=TeX,Scale=1}
\fi
% Use upquote if available, for straight quotes in verbatim environments
\IfFileExists{upquote.sty}{\usepackage{upquote}}{}
\IfFileExists{microtype.sty}{% use microtype if available
  \usepackage[]{microtype}
  \UseMicrotypeSet[protrusion]{basicmath} % disable protrusion for tt fonts
}{}
\makeatletter
\@ifundefined{KOMAClassName}{% if non-KOMA class
  \IfFileExists{parskip.sty}{%
    \usepackage{parskip}
  }{% else
    \setlength{\parindent}{0pt}
    \setlength{\parskip}{6pt plus 2pt minus 1pt}}
}{% if KOMA class
  \KOMAoptions{parskip=half}}
\makeatother
\usepackage{xcolor}
\setlength{\emergencystretch}{3em} % prevent overfull lines
\setcounter{secnumdepth}{5}
% Make \paragraph and \subparagraph free-standing
\ifx\paragraph\undefined\else
  \let\oldparagraph\paragraph
  \renewcommand{\paragraph}[1]{\oldparagraph{#1}\mbox{}}
\fi
\ifx\subparagraph\undefined\else
  \let\oldsubparagraph\subparagraph
  \renewcommand{\subparagraph}[1]{\oldsubparagraph{#1}\mbox{}}
\fi

\usepackage{color}
\usepackage{fancyvrb}
\newcommand{\VerbBar}{|}
\newcommand{\VERB}{\Verb[commandchars=\\\{\}]}
\DefineVerbatimEnvironment{Highlighting}{Verbatim}{commandchars=\\\{\}}
% Add ',fontsize=\small' for more characters per line
\newenvironment{Shaded}{}{}
\newcommand{\AlertTok}[1]{\textcolor[rgb]{1.00,0.00,0.00}{\textbf{#1}}}
\newcommand{\AnnotationTok}[1]{\textcolor[rgb]{0.38,0.63,0.69}{\textbf{\textit{#1}}}}
\newcommand{\AttributeTok}[1]{\textcolor[rgb]{0.49,0.56,0.16}{#1}}
\newcommand{\BaseNTok}[1]{\textcolor[rgb]{0.25,0.63,0.44}{#1}}
\newcommand{\BuiltInTok}[1]{#1}
\newcommand{\CharTok}[1]{\textcolor[rgb]{0.25,0.44,0.63}{#1}}
\newcommand{\CommentTok}[1]{\textcolor[rgb]{0.38,0.63,0.69}{\textit{#1}}}
\newcommand{\CommentVarTok}[1]{\textcolor[rgb]{0.38,0.63,0.69}{\textbf{\textit{#1}}}}
\newcommand{\ConstantTok}[1]{\textcolor[rgb]{0.53,0.00,0.00}{#1}}
\newcommand{\ControlFlowTok}[1]{\textcolor[rgb]{0.00,0.44,0.13}{\textbf{#1}}}
\newcommand{\DataTypeTok}[1]{\textcolor[rgb]{0.56,0.13,0.00}{#1}}
\newcommand{\DecValTok}[1]{\textcolor[rgb]{0.25,0.63,0.44}{#1}}
\newcommand{\DocumentationTok}[1]{\textcolor[rgb]{0.73,0.13,0.13}{\textit{#1}}}
\newcommand{\ErrorTok}[1]{\textcolor[rgb]{1.00,0.00,0.00}{\textbf{#1}}}
\newcommand{\ExtensionTok}[1]{#1}
\newcommand{\FloatTok}[1]{\textcolor[rgb]{0.25,0.63,0.44}{#1}}
\newcommand{\FunctionTok}[1]{\textcolor[rgb]{0.02,0.16,0.49}{#1}}
\newcommand{\ImportTok}[1]{#1}
\newcommand{\InformationTok}[1]{\textcolor[rgb]{0.38,0.63,0.69}{\textbf{\textit{#1}}}}
\newcommand{\KeywordTok}[1]{\textcolor[rgb]{0.00,0.44,0.13}{\textbf{#1}}}
\newcommand{\NormalTok}[1]{#1}
\newcommand{\OperatorTok}[1]{\textcolor[rgb]{0.40,0.40,0.40}{#1}}
\newcommand{\OtherTok}[1]{\textcolor[rgb]{0.00,0.44,0.13}{#1}}
\newcommand{\PreprocessorTok}[1]{\textcolor[rgb]{0.74,0.48,0.00}{#1}}
\newcommand{\RegionMarkerTok}[1]{#1}
\newcommand{\SpecialCharTok}[1]{\textcolor[rgb]{0.25,0.44,0.63}{#1}}
\newcommand{\SpecialStringTok}[1]{\textcolor[rgb]{0.73,0.40,0.53}{#1}}
\newcommand{\StringTok}[1]{\textcolor[rgb]{0.25,0.44,0.63}{#1}}
\newcommand{\VariableTok}[1]{\textcolor[rgb]{0.10,0.09,0.49}{#1}}
\newcommand{\VerbatimStringTok}[1]{\textcolor[rgb]{0.25,0.44,0.63}{#1}}
\newcommand{\WarningTok}[1]{\textcolor[rgb]{0.38,0.63,0.69}{\textbf{\textit{#1}}}}

\providecommand{\tightlist}{%
  \setlength{\itemsep}{0pt}\setlength{\parskip}{0pt}}\usepackage{longtable,booktabs,array}
\usepackage{calc} % for calculating minipage widths
% Correct order of tables after \paragraph or \subparagraph
\usepackage{etoolbox}
\makeatletter
\patchcmd\longtable{\par}{\if@noskipsec\mbox{}\fi\par}{}{}
\makeatother
% Allow footnotes in longtable head/foot
\IfFileExists{footnotehyper.sty}{\usepackage{footnotehyper}}{\usepackage{footnote}}
\makesavenoteenv{longtable}
\usepackage{graphicx}
\makeatletter
\def\maxwidth{\ifdim\Gin@nat@width>\linewidth\linewidth\else\Gin@nat@width\fi}
\def\maxheight{\ifdim\Gin@nat@height>\textheight\textheight\else\Gin@nat@height\fi}
\makeatother
% Scale images if necessary, so that they will not overflow the page
% margins by default, and it is still possible to overwrite the defaults
% using explicit options in \includegraphics[width, height, ...]{}
\setkeys{Gin}{width=\maxwidth,height=\maxheight,keepaspectratio}
% Set default figure placement to htbp
\makeatletter
\def\fps@figure{htbp}
\makeatother
\newlength{\cslhangindent}
\setlength{\cslhangindent}{1.5em}
\newlength{\csllabelwidth}
\setlength{\csllabelwidth}{3em}
\newlength{\cslentryspacingunit} % times entry-spacing
\setlength{\cslentryspacingunit}{\parskip}
\newenvironment{CSLReferences}[2] % #1 hanging-ident, #2 entry spacing
 {% don't indent paragraphs
  \setlength{\parindent}{0pt}
  % turn on hanging indent if param 1 is 1
  \ifodd #1
  \let\oldpar\par
  \def\par{\hangindent=\cslhangindent\oldpar}
  \fi
  % set entry spacing
  \setlength{\parskip}{#2\cslentryspacingunit}
 }%
 {}
\usepackage{calc}
\newcommand{\CSLBlock}[1]{#1\hfill\break}
\newcommand{\CSLLeftMargin}[1]{\parbox[t]{\csllabelwidth}{#1}}
\newcommand{\CSLRightInline}[1]{\parbox[t]{\linewidth - \csllabelwidth}{#1}\break}
\newcommand{\CSLIndent}[1]{\hspace{\cslhangindent}#1}

\usepackage{booktabs}
\usepackage{caption}
\usepackage{longtable}
\usepackage{colortbl}
\usepackage{array}
\KOMAoption{captions}{tableheading}
\makeatletter
\@ifpackageloaded{tcolorbox}{}{\usepackage[many]{tcolorbox}}
\@ifpackageloaded{fontawesome5}{}{\usepackage{fontawesome5}}
\definecolor{quarto-callout-color}{HTML}{909090}
\definecolor{quarto-callout-note-color}{HTML}{0758E5}
\definecolor{quarto-callout-important-color}{HTML}{CC1914}
\definecolor{quarto-callout-warning-color}{HTML}{EB9113}
\definecolor{quarto-callout-tip-color}{HTML}{00A047}
\definecolor{quarto-callout-caution-color}{HTML}{FC5300}
\definecolor{quarto-callout-color-frame}{HTML}{acacac}
\definecolor{quarto-callout-note-color-frame}{HTML}{4582ec}
\definecolor{quarto-callout-important-color-frame}{HTML}{d9534f}
\definecolor{quarto-callout-warning-color-frame}{HTML}{f0ad4e}
\definecolor{quarto-callout-tip-color-frame}{HTML}{02b875}
\definecolor{quarto-callout-caution-color-frame}{HTML}{fd7e14}
\makeatother
\makeatletter
\makeatother
\makeatletter
\makeatother
\makeatletter
\@ifpackageloaded{caption}{}{\usepackage{caption}}
\AtBeginDocument{%
\ifdefined\contentsname
  \renewcommand*\contentsname{Índice}
\else
  \newcommand\contentsname{Índice}
\fi
\ifdefined\listfigurename
  \renewcommand*\listfigurename{Lista de Figuras}
\else
  \newcommand\listfigurename{Lista de Figuras}
\fi
\ifdefined\listtablename
  \renewcommand*\listtablename{Lista de Tabelas}
\else
  \newcommand\listtablename{Lista de Tabelas}
\fi
\ifdefined\figurename
  \renewcommand*\figurename{Figura}
\else
  \newcommand\figurename{Figura}
\fi
\ifdefined\tablename
  \renewcommand*\tablename{Tabela}
\else
  \newcommand\tablename{Tabela}
\fi
}
\@ifpackageloaded{float}{}{\usepackage{float}}
\floatstyle{ruled}
\@ifundefined{c@chapter}{\newfloat{codelisting}{h}{lop}}{\newfloat{codelisting}{h}{lop}[chapter]}
\floatname{codelisting}{Listagem}
\newcommand*\listoflistings{\listof{codelisting}{Lista de Listagens}}
\makeatother
\makeatletter
\@ifpackageloaded{caption}{}{\usepackage{caption}}
\@ifpackageloaded{subcaption}{}{\usepackage{subcaption}}
\makeatother
\makeatletter
\makeatother
\ifLuaTeX
\usepackage[bidi=basic]{babel}
\else
\usepackage[bidi=default]{babel}
\fi
\babelprovide[main,import]{brazilian}
% get rid of language-specific shorthands (see #6817):
\let\LanguageShortHands\languageshorthands
\def\languageshorthands#1{}
\ifLuaTeX
  \usepackage{selnolig}  % disable illegal ligatures
\fi
\IfFileExists{bookmark.sty}{\usepackage{bookmark}}{\usepackage{hyperref}}
\IfFileExists{xurl.sty}{\usepackage{xurl}}{} % add URL line breaks if available
\urlstyle{same} % disable monospaced font for URLs
\hypersetup{
  pdftitle={Tutorial R Módulo 1. Importando planilhas do Excel},
  pdfauthor={Prof.~Elvio S. F. Medeiros; Laboratório de Ecologia; Universidade Estadual da Paraíba; Campus V, João Pessoa, PB},
  pdflang={pt-br},
  colorlinks=true,
  linkcolor={blue},
  filecolor={Maroon},
  citecolor={Blue},
  urlcolor={Blue},
  pdfcreator={LaTeX via pandoc}}

\title{Tutorial R Módulo 1. Importando planilhas do Excel}
\usepackage{etoolbox}
\makeatletter
\providecommand{\subtitle}[1]{% add subtitle to \maketitle
  \apptocmd{\@title}{\par {\large #1 \par}}{}{}
}
\makeatother
\subtitle{Disciplina de Ecologia Numérica\footnote{Curso de Ciências
  Biológicas do Campus V da UEPB}}
\author{Prof.~Elvio S. F. Medeiros \and Laboratório de
Ecologia \and Universidade Estadual da Paraíba \and Campus V, João
Pessoa, PB}
\date{17/01/2026}

\begin{document}
\maketitle
\begin{abstract}
A importação de planilhas do Excel para o ambiente de programação R é
uma tarefa fundamental para análise de dados e estatísticas. Através da
importação de planilhas do Excel, é possível transformar dados
armazenados em formatos familiares em estruturas que podem ser
manipuladas e exploradas de maneira eficaz no R. Isso permite a
aplicação de diversas técnicas estatísticas e criação de visualizações
informativas, contribuindo para a tomada de decisões embasadas em dados.
Neste contexto, entender como importar dados do Excel para o R é um
passo crucial para realizar análises de alta qualidade e obter insights
significativos a partir dos conjuntos de dados disponíveis.
\end{abstract}
\renewcommand*\contentsname{Índice}
{
\hypersetup{linkcolor=}
\setcounter{tocdepth}{3}
\tableofcontents
}
\listoffigures
\listoftables
\hypertarget{apresentauxe7uxe3o}{%
\section{Apresentação}\label{apresentauxe7uxe3o}}

A importação de planilhas do Excel para o ambiente de programação R é
uma tarefa fundamental para análise de dados e estatísticas. O R é uma
linguagem de programação amplamente utilizada por cientistas de dados,
pesquisadores e analistas para manipular, visualizar e modelar
informações. Através da importação de planilhas do Excel, é possível
transformar dados armazenados em formatos familiares em estruturas que
podem ser manipuladas e exploradas de maneira eficaz no R. Isso permite
a aplicação de diversas técnicas estatísticas e criação de visualizações
informativas, contribuindo para a tomada de decisões embasadas em dados.
Neste contexto, entender como importar dados do Excel para o R é um
passo crucial para realizar análises de alta qualidade e obter insights
significativos a partir dos conjuntos de dados disponíveis.

\hypertarget{instalauxe7uxe3o-do-r-e-rstudio}{%
\section{Instalação do R e
RStudio}\label{instalauxe7uxe3o-do-r-e-rstudio}}

\hypertarget{r-base}{%
\subsection{R base}\label{r-base}}

\begin{enumerate}
\def\labelenumi{\arabic{enumi}.}
\tightlist
\item
  O primeiro passo é entrar na página do projeto CRAN (Comprehensive R
  Archive Network) \url{https://www.r-project.org/}.\\
\item
  Do lado esquerdo da página clique sobre o link CRAN abaixo de
  Download. 3. Uma nova página com uma série de links irá se abrir.
  Esses links são chamados de ``espelhos'' e servem para que você possa
  escolher o local mais próximo de onde você está para fazer o download
  do programa. Escolha um espelho no Brasil.\\
\item
  Na seção Download and Install R, clique sobre o link
  \texttt{Download\ R\ for\ Windows} para baixar a versão para esse
  sistema\ldots{} (MacOS??\ldots{} o que é isso?)\\
\item
  Clique sobre o link \texttt{base}.
\item
  Clique sobre o link \texttt{Download\ R\ 4.x.x\ for\ Windows} para
  fazer o download do arquivo R.exe.
\item
  A instalação segue o formato padrão de instalação de programas no
  Windows, e portanto não são necessários maiores detalhes.
\end{enumerate}

\hypertarget{rstudio}{%
\subsection{RStudio}\label{rstudio}}

\begin{enumerate}
\def\labelenumi{\arabic{enumi}.}
\tightlist
\item
  Para baixar o RStudio entre no endereço
  \url{https://posit.co/download/rstudio-desktop/}\\
\item
  Clique no link \texttt{Products\ \textgreater{}\ RStudio}\\
\item
  Selecione a versão Desktop.\\
\item
  Clique em \texttt{DOWNLOAD\ RSTUDIO\ DESKTOP}\\
\item
  Será exibida uma página com a recomendação para você baixar o RStudio
  FREE versão mais recente - Windows\\
\item
  Clicando nesse link, você irá baixar o arquivo RStudio atual (.exe)
\item
  Depois é só clicar e instalar da forma convencional do Windows.
\end{enumerate}

Após a instalação, você pode abrir o \texttt{RStudio} pelo seu
respectivo ícone, e o RStudio estará pronto para ser utilizado. O ``R
base'' continuará instalado mas será acessado pelo RStudio. Não o
desinstale.

\hypertarget{rstudio-na-nuvem}{%
\section{RStudio na nuvem}\label{rstudio-na-nuvem}}

Usar o RStudio Cloud \url{https://login.rstudio.cloud/} é uma opção para
quem não quer instalar a versão para PC. O RStudio Cloud é uma
plataforma online que fornece um ambiente de desenvolvimento integrado
para o R, permitindo que os usuários executem análises, desenvolvam
código e colaborem com outras pessoas, sem a necessidade de instalar o R
e o RStudio em seus próprios computadores. É uma solução conveniente e
acessível, especialmente para iniciantes ou usuários que desejam
compartilhar projetos e colaborar de forma eficiente.

\hypertarget{sobre-os-dados-do-ppbio}{%
\subsection{Sobre os dados do PPBio}\label{sobre-os-dados-do-ppbio}}

Usaremos para esse tutorial dados coletados no Programa de Pesquisa em
Biodiversidade - PPBio (veja
\href{https://ppbio.inpa.gov.br/Sobre}{Programa de Pesquisa em
Biodiversidade -- PPBio}). Nesta base de dados estão armazenadas
informações sobre diversos grupos taxonômicos dstribuidos em diversas
unidades amostrais (UA's ou sítios), como peixes, macroinverbrebrados
bentônicos, quironomídeos e zooplâncotn, além de dados do habitat, como
variáveis físicas e químicas, morfologia do habitat, composição do
substrato, estrutura de habitat marginal, entre outros
(Figura~\ref{fig-ppbio})). Essa é a \textbf{matriz bruta de dados},
porque os valores ainda não foram ajustados para os valores de Captura
Por Unidade de Esforço (CPUE), nem foram relativizados ou transformados
(Tabela~\ref{tbl-301arqs}).

\begin{figure}

{\centering \includegraphics{D:/Elvio/OneDrive/Disciplinas/_EcoNumerica/3.R_Quarto/imagens/ppbio.png}

}

\caption{\label{fig-ppbio}Parte da planilha de dados brutos do PPBio.}

\end{figure}

\hypertarget{tbl-301arqs}{}
\begin{longtable}{>{\raggedright\arraybackslash}
\caption{\label{tbl-301arqs}Matrizes disponíveis para análises, com suas descrições e tipos de dados
recomendados. }\tabularnewline
p{100px}>{\raggedright\arraybackslash}p{100px}>{\raggedright\arraybackslash}p{250px}>{\raggedright\arraybackslash}p{200px}}
\toprule
Arquivo (.xlsx) & Tipo de matriz & Descrição & Tipo de dados \\ 
\midrule\addlinespace[2.5pt]
\href{https://docs.google.com/spreadsheets/d/1Olyew6L3SCFeSXWJC54Ak3MKmgHD3sTK/edit?usp=drive_link\&ouid=101766125969589673518\&rtpof=true\&sd=true}{ppbio06c-peixes} & Matriz comunitária & O arquivo ppbio06 traz os dados brutos que serão usados nas análises. A matriz de dados brutos contendo 26 localidades em estações do ano diferentes (objetos) x 35 espécies (atributos), antes de qualquer modificação. & Contagens de indivíduos com alta amplitude de variação, sugerido uso de matriz relativizada. \\ 
\href{https://docs.google.com/spreadsheets/d/1V_QxcorksaSOa40Uzj1z2jy02WgKeFTx/edit?usp=drive_link\&ouid=101766125969589673518\&rtpof=true\&sd=true}{ppbio06p-amb} & Matriz ambiental & O arquivo ppbio06h traz os dados brutos que serão usados nas análises. A matriz de dados brutos contendo 26 localidades em estações diferentes (objetos) x 35 variáveis ambientais (atributos) medidas em diferentes escalas espaciais, antes de qualquer modificação. & Unidades de medição diferentes (cm, m, °C, mg/L, etc.), com alta amplitude de variação, sugerido uso de matriz transformada e/ou reescalada. \\ 
\href{https://docs.google.com/spreadsheets/d/1H52eBBxZ1k-6Xgrse87etq6mq3mMUTZi/edit?usp=drive_link\&ouid=101766125969589673518\&rtpof=true\&sd=true}{ppbio06-grupos} & Matriz de grupos & O arquivo ppbio06 traz os dados brutos que serão usados nessa análise. A matriz de dados brutos contendo 26 locais/ocasiões (objetos) x 35 espécies (atributos), antes de qualquer modificação. & Contagens de indivíduos com alta amplitude de variação, sugerido uso de matriz relativizada. \\ 
ppbio06cpue & Matriz comunitária & O arquivo ppbio06cpue traz os valores após ajuste pela Captura Por Unidade de Esforço (CPUE). & Densidades de indivíduos com alta amplitude de variação, sugerido uso de matriz relativizada. \\ 
\bottomrule
\end{longtable}

A planilha \texttt{ppbio} contém o delineamento amostral de um dos
estudos do Projeto PPBio (Figura Figura~\ref{fig-fig3}). Nas linhas são
apresentadas as abreviações dos nomes das unidades amostrais (UA's) e
nas colunas são apresentados os nomes abreviados das espécies - temos
portando uma matriz comunitária (Tabela~\ref{tbl-301arqs})). No corpo da
planilha temos os valores para o tipo de dados amostrado. Quantitativo,
semi-quatitativo ou qualitativo.\\
Qual desses tipos de dados você acha que é apresentado na planilha?

\begin{figure}

{\centering \includegraphics{D:/Elvio/OneDrive/Disciplinas/_EcoNumerica/3.R_Quarto/imagens/rm1.fig3.png}

}

\caption{\label{fig-fig3}Associação entre a planilha de dados brutos do
PPBio e o delineamento amostral do estudo.}

\end{figure}

Várias das espécies nessa matriz tem grande importância ecológica, como
é o caso de \emph{Astyanax bimaculatus} \footnote{A etimologia do gênero
  \emph{Astyanax} vem da mitologia Grega. Heitor personagem da
  ``Ilíada'', tinha um filho chamado Astíanax.}
(Figura~\ref{fig-asbim}), que é muito comum em rios intermitentes e
serve de alimento para predadores maiores como a espécie \emph{Hoplias
malabaricus} \footnote{Do Grego, \emph{hoplon}, arma ou armadura, em
  referência aos dentes caniniformes muito desenvolvidos, e forte
  estrutura óssea na cabeça.} (Figura~\ref{fig-hmala}).

\begin{figure}

{\centering \includegraphics{D:/Elvio/OneDrive/Disciplinas/_EcoNumerica/3.R_Quarto/imagens/spp/Asbim_u4.jpg}

}

\caption{\label{fig-asbim}\emph{Astyanax bimaculatus}, a espécie mais
comum da matriz de dados ppbio. Peru, by Eakins, R. Fonte:
\url{https://www.fishbase.se/summary/Astianax-bimaculatus.html}}

\end{figure}

\begin{figure}

{\centering \includegraphics{D:/Elvio/OneDrive/Disciplinas/_EcoNumerica/3.R_Quarto/imagens/spp/Homal_u6.jpg}

}

\caption{\label{fig-hmala}\emph{Hoplias malabaricus}, espécie que cresce
para se tornar um importante predador. Brazil, by Roselet, F.F.G. Fonte:
\url{https://www.fishbase.se/summary/Hoplias-malabaricus.html}}

\end{figure}

\hypertarget{importando-a-planilha-de-trabalho}{%
\section{Importando a planilha de
trabalho}\label{importando-a-planilha-de-trabalho}}

Para começar a usar o R e analisar os dados do Projeto PPBio, abra o
RStudio, verifique sua interface (Figura~\ref{fig-interface})) e siga as
instruções a seguir.

\begin{figure}

{\centering \includegraphics{D:/Elvio/OneDrive/Disciplinas/_EcoNumerica/3.R_Quarto/imagens/interfaceR.png}

}

\caption{\label{fig-interface}Interface típica do RStudio e nome dos
paineis ou janelas.}

\end{figure}

\begin{tcolorbox}[enhanced jigsaw, left=2mm, toprule=.15mm, leftrule=.75mm, colback=white, coltitle=black, opacityback=0, colframe=quarto-callout-caution-color-frame, breakable, arc=.35mm, bottomtitle=1mm, colbacktitle=quarto-callout-caution-color!10!white, toptitle=1mm, opacitybacktitle=0.6, title=\textcolor{quarto-callout-caution-color}{\faFire}\hspace{0.5em}{Mensagens de erro e avisos no R}, rightrule=.15mm, titlerule=0mm, bottomrule=.15mm]

No contexto da linguagem de programação R, mensagens de erro (errors) e
mensagens de aviso (warnings) que {aparecem em vermelho no painel de
console}. Elas são formas de feedback do sistema que indicam problemas
ou situações potencialmente problemáticas durante a execução do código.
Aqui está uma breve explicação de cada um:

\begin{enumerate}
\def\labelenumi{\arabic{enumi}.}
\tightlist
\item
  \textbf{Erro ({Error}):}

  \begin{itemize}
  \tightlist
  \item
    Um erro ocorre quando algo no código não está correto ou não pode
    ser executado como esperado.
  \item
    Isso pode ser causado por sintaxe incorreta, uso incorreto de
    funções, operações inválidas, referências a objetos que não existem,
    entre outros problemas.
  \item
    Quando ocorre um erro, a execução do código é interrompida e uma
    mensagem de erro é exibida no console em vermelho, indicando o tipo
    de erro e, muitas vezes, a linha onde ocorreu.
  \end{itemize}
\item
  \textbf{Aviso ({Warning}):}

  \begin{itemize}
  \tightlist
  \item
    Não indica erro. Um aviso é emitido quando algo no código pode
    resultar em um comportamento indesejado ou em resultados
    inesperados, mas não interrompe necessariamente a execução do
    código.
  \item
    Os avisos geralmente indicam situações que merecem atenção, como
    conversões de tipos de dados que podem perder informações ou funções
    que estão sendo usadas de maneira que pode levar a resultados
    questionáveis.
  \item
    Os avisos são exibidos em vermelho no console e fornecem informações
    sobre a natureza do aviso e, possivelmente, como abordá-lo.
  \end{itemize}
\end{enumerate}

É importante prestar atenção a mensagens de erro e avisos, pois eles
fornecem insights sobre problemas em seu código ou potenciais fontes de
comportamento inesperado. Resolver erros é fundamental para que o código
funcione conforme o esperado. Embora os avisos não interrompam a
execução, investigá-los pode ajudar a evitar problemas futuros ou
melhorar a qualidade do código.

\end{tcolorbox}

\hypertarget{organizauxe7uxe3o-buxe1sica}{%
\subsection{Organização básica}\label{organizauxe7uxe3o-buxe1sica}}

\begin{tcolorbox}[enhanced jigsaw, left=2mm, toprule=.15mm, leftrule=.75mm, colback=white, coltitle=black, opacityback=0, colframe=quarto-callout-color-frame, breakable, arc=.35mm, bottomtitle=1mm, colbacktitle=quarto-callout-color!10!white, toptitle=1mm, opacitybacktitle=0.6, title={Attention}, rightrule=.15mm, titlerule=0mm, bottomrule=.15mm]

Os links para baixar as planilhas necessárias para repetir esse tutorial
podem ser encontrados na seção @ref(download401)

\end{tcolorbox}

No ambiente do RStudio no painel de edição de código execute
(\texttt{Ctrl+Enter} com o teclado ou \texttt{Run} no editor de código)
os comandos a seguir, para instalar os pacotes necessários para este
módulo.

\begin{Shaded}
\begin{Highlighting}[numbers=left,,]
\FunctionTok{install.packages}\NormalTok{(}\StringTok{"readxl"}\NormalTok{) }\CommentTok{\#importa arquivos do excel}
\end{Highlighting}
\end{Shaded}

E em seguida,

\begin{Shaded}
\begin{Highlighting}[numbers=left,,]
\FunctionTok{library}\NormalTok{(readxl)}
\end{Highlighting}
\end{Shaded}

Os códigos acima, são usados para instalar e carregar os pacotes
necessários para este módulo. Esses códigos são comandos para instalar
pacotes no R. Um pacote é uma coleção de funções, dados e documentação
que ampliam as capacidades do R (\href{https://cran.r-project.org/}{R
CRAN}) (R Core Team 2017), e
\href{https://posit.co/download/rstudio-desktop/}{RStudio} (Team 2022).
No exemplo acima, o pacote \texttt{readxl} permite ler e escrever
arquivos Excel no R.

Para instalar um pacote no R, você precisa usar a função
\texttt{install.packages()}. Depois de instalar um pacote, você precisa
carregá-lo na sua sessão R com a função \texttt{library()}.

Por exemplo, para carregar o pacote \texttt{readxl}, você precisa
executar a função \texttt{library(readxl)}. Isso irá permitir que você
use as funções do pacote na sua sessão R. Você precisa carregar um
pacote toda vez que iniciar uma nova sessão R e quiser usar um pacote
instalado.

Agora vamos \textbf{definir o diretório de trabalho}. Esse código é
usado para obter e definir o diretório de trabalho atual no R. O comando
\texttt{getwd()} retorna o caminho do diretório onde o R está lendo e
salvando arquivos. O comando \texttt{setwd()} muda esse diretório de
trabalho para o caminho especificado entre aspas. No seu caso, você deve
ajustar o caminho para o seu próprio diretório de trabalho.
\textbf{Lembre de usar a barra ``/'' entre os diretórios. E não a
contra-barra ``\textbackslash{}''.}

Usaremos uma matriz multivariada (sítios x espécies, matriz comunitária)
do Projeto PPBio chamada ppbio**.xlsx que está no diretório
{``C:/Meu/Diretório/De/Trabalho/Planilha.xlsx''}

Note que o símbolo \texttt{\#} em programação R significa que o texto
que vem depois dele é um comentário e não será executado pelo programa.
Isso é útil para explicar o código ou deixar anotações.

Ajuste a segunda linha do código abaixo para refletir
{``C:/Seu/Diretório/De/Trabalho/Planilha.xlsx''}.

Definindo o diretório de trabalho e installando os pacotes necessários:

\begin{Shaded}
\begin{Highlighting}[numbers=left,,]
\FunctionTok{getwd}\NormalTok{()}
\FunctionTok{setwd}\NormalTok{(}\StringTok{"C:/Seu/Diretório/De/Trabalho"}\NormalTok{)}
\end{Highlighting}
\end{Shaded}

Alternativamente você pode ir na barra de tarefas e escolhes as
opções:\textbackslash SESSION -\textgreater{} SET WORKING DIRECTORY
-\textgreater{} CHOOSE DIRECTORY

\hypertarget{prefira-sempre-cuxf3digos-e-scripts-do-que-mouse-e-menus-de-janelas-no-r}{%
\subsubsection{Prefira sempre códigos e scripts do que mouse e menus de
janelas no
R}\label{prefira-sempre-cuxf3digos-e-scripts-do-que-mouse-e-menus-de-janelas-no-r}}

\begin{tcolorbox}[enhanced jigsaw, left=2mm, toprule=.15mm, leftrule=.75mm, colback=white, coltitle=black, opacityback=0, colframe=quarto-callout-caution-color-frame, breakable, arc=.35mm, bottomtitle=1mm, colbacktitle=quarto-callout-caution-color!10!white, toptitle=1mm, opacitybacktitle=0.6, title=\textcolor{quarto-callout-caution-color}{\faFire}\hspace{0.5em}{Porque preferir códigos e scripts do que mouse e menus de janelas no R}, rightrule=.15mm, titlerule=0mm, bottomrule=.15mm]

Optar pelo uso de scripts e comandos de teclado no R, em vez das opções
baseadas em mouse e menus das janelas, oferece várias vantagens
significativas para quem está envolvido em análises de dados e
programação. Aqui estão algumas justificativas para essa abordagem:

\begin{enumerate}
\def\labelenumi{\arabic{enumi}.}
\item
  \textbf{Reprodutibilidade:} O uso de scripts permite que todas as
  etapas de análise e manipulação de dados sejam documentadas em um
  único lugar. Isso facilita a reexecução de todo o processo, tornando a
  análise reprodutível e permitindo que outras pessoas compreendam e
  validem o trabalho realizado.
\item
  \textbf{Automação:} Comandos de script podem ser facilmente repetidos
  ou adaptados para diferentes conjuntos de dados. Isso possibilita a
  automação de tarefas complexas, economizando tempo e reduzindo a
  possibilidade de erros manuais.
\item
  \textbf{Flexibilidade:} Enquanto as opções de mouse e menus podem ser
  limitadas em termos das ações específicas que permitem, os scripts
  oferecem uma flexibilidade muito maior. Você pode personalizar cada
  etapa do processo de análise de acordo com suas necessidades
  específicas.
\item
  \textbf{Eficiência:} A digitação de comandos é geralmente mais rápida
  do que navegar por menus e clicar em botões, especialmente quando se
  trata de tarefas repetitivas e/ou complexas.
\item
  \textbf{Controle total:} Ao utilizar scripts, você tem controle total
  sobre cada etapa do processo. Isso é particularmente importante em
  análises estatísticas, onde pequenas variações nos parâmetros podem
  ter um grande impacto nos resultados.
\item
  \textbf{Aprendizado contínuo:} Escrever e modificar scripts permite um
  maior aprendizado e domínio da linguagem R. Conforme você ganha
  experiência, poderá realizar análises mais sofisticadas e explorar
  recursos avançados.
\item
  \textbf{Portabilidade:} Scripts podem ser facilmente compartilhados
  com outros pesquisadores ou colegas, independentemente do sistema
  operacional utilizado. Isso torna a colaboração mais fluida e ajuda a
  evitar problemas de compatibilidade.
\item
  \textbf{Melhor entendimento:} Ao escrever e ler scripts, você
  desenvolve uma compreensão mais profunda dos processos que está
  realizando. Isso é importante para identificar possíveis erros e
  interpretar corretamente os resultados.
\item
  \textbf{Documentação clara:} Ao escrever um script, você pode
  adicionar comentários explicativos que descrevem cada passo e sua
  lógica. Isso resulta em uma documentação clara e autoexplicativa do
  trabalho realizado.
\item
  \textbf{Consistência:} O uso de scripts promove a adoção de práticas
  consistentes em toda a análise, reduzindo a chance de erros causados
  por abordagens diferentes em momentos distintos.
\end{enumerate}

Em resumo, a abordagem baseada em scripts e comandos de teclado oferece
mais controle, flexibilidade, eficiência e reprodutibilidade, tornando-a
a escolha preferida para profissionais que buscam análises de dados
precisas, consistentes e de alta qualidade no ambiente R.

\end{tcolorbox}

\hypertarget{importando-a-planilha}{%
\section{Importando a planilha}\label{importando-a-planilha}}

\begin{Shaded}
\begin{Highlighting}[numbers=left,,]
\FunctionTok{library}\NormalTok{(readxl)}
\NormalTok{ppbio06 }\OtherTok{\textless{}{-}} \FunctionTok{read\_excel}\NormalTok{(}\StringTok{"D:/Elvio/OneDrive/Disciplinas/\_EcoNumerica/5.Matrizes/ppbio06.xlsx"}\NormalTok{, }\AttributeTok{sheet =} \StringTok{"Sheet1"}\NormalTok{, }\AttributeTok{na =} \StringTok{"NA"}\NormalTok{)}
\FunctionTok{str}\NormalTok{(ppbio06)}
\FunctionTok{class}\NormalTok{(ppbio06)}
\end{Highlighting}
\end{Shaded}

Com essas linhas de código a primeira coluna da matriz importada
apresenta texto. Não queremos assim porque vamos fazer cálculos
matemáticos na matriz.

Resolvemos o problema com mais algumas linhas de código.

\begin{Shaded}
\begin{Highlighting}[numbers=left,,]
\NormalTok{ppbio06 }\OtherTok{\textless{}{-}} \FunctionTok{as.data.frame}\NormalTok{(ppbio06)}
\FunctionTok{class}\NormalTok{(ppbio06)}
\FunctionTok{rownames}\NormalTok{(ppbio06) }\OtherTok{\textless{}{-}}\NormalTok{ ppbio06[,}\DecValTok{1}\NormalTok{] }\CommentTok{\#tem  que ser um df}
\NormalTok{ppbio06[,}\DecValTok{1}\NormalTok{] }\OtherTok{\textless{}{-}} \ConstantTok{NULL}
\end{Highlighting}
\end{Shaded}

Ou podemos instalar esse pacote de importação de arquivos .xlsx para o
R.

\begin{Shaded}
\begin{Highlighting}[numbers=left,,]
\FunctionTok{install.packages}\NormalTok{(}\StringTok{"openxlsx"}\NormalTok{)}
\end{Highlighting}
\end{Shaded}

Carregamos o pacote \texttt{openxlsx}

\begin{Shaded}
\begin{Highlighting}[numbers=left,,]
\FunctionTok{library}\NormalTok{(openxlsx)}
\end{Highlighting}
\end{Shaded}

\begin{verbatim}
Warning: package 'openxlsx' was built under R version 4.3.2
\end{verbatim}

Importamos novamente a planilha, usando esse novo pacote.

\begin{Shaded}
\begin{Highlighting}[numbers=left,,]
\NormalTok{ppbio }\OtherTok{\textless{}{-}} \FunctionTok{read.xlsx}\NormalTok{(}\StringTok{"D:/Elvio/OneDrive/Disciplinas/\_EcoNumerica/5.Matrizes/ppbio06.xlsx"}\NormalTok{,}
                   \AttributeTok{rowNames =}\NormalTok{ T,}
                   \AttributeTok{colNames =}\NormalTok{ T,}
                   \AttributeTok{sheet =} \StringTok{"Sheet1"}\NormalTok{)}
\FunctionTok{str}\NormalTok{(ppbio)}
\FunctionTok{class}\NormalTok{(ppbio)}
\NormalTok{ppbio\_ma }\OtherTok{\textless{}{-}} \FunctionTok{as.matrix}\NormalTok{(ppbio) }\CommentTok{\#lê ppbio como uma matriz}
\FunctionTok{str}\NormalTok{(ppbio\_ma)}
\FunctionTok{class}\NormalTok{(ppbio\_ma)}
\CommentTok{\#ppbio}
\CommentTok{\#ppbio\_ma}
\end{Highlighting}
\end{Shaded}

Compare as diferenças. Agora podemos exportar os dados como uma matriz
de dados em formato de
\texttt{valores\ separados\ por\ vírgula\ (.csv)}.

\begin{Shaded}
\begin{Highlighting}[numbers=left,,]
\FunctionTok{write.table}\NormalTok{(ppbio, }\StringTok{"ppbiocsv.txt"}\NormalTok{, }\AttributeTok{append =}\NormalTok{ F, }\AttributeTok{quote =}\NormalTok{ T, }\StringTok{";"}\NormalTok{, }\AttributeTok{row.names =}\NormalTok{ T)}
\NormalTok{dir }\OtherTok{\textless{}{-}} \FunctionTok{getwd}\NormalTok{()}
\FunctionTok{shell.exec}\NormalTok{(dir) }\CommentTok{\#abre o diretorio de trabalho no Windows Explorer}
\end{Highlighting}
\end{Shaded}

Podemos carregar o arquivo .csv criado \texttt{ppbiocsv.txt} usando os
códigos abaixo.

\begin{Shaded}
\begin{Highlighting}[numbers=left,,]
\NormalTok{ppbiocsv }\OtherTok{\textless{}{-}} \FunctionTok{read.csv}\NormalTok{(}\StringTok{"ppbiocsv.txt"}\NormalTok{,}
                     \AttributeTok{sep =} \StringTok{";"}\NormalTok{, }\AttributeTok{dec =} \StringTok{","}\NormalTok{, }\CommentTok{\#definimos o dígito separador}
                     \AttributeTok{header =}\NormalTok{ T,}
                     \AttributeTok{row.names =} \DecValTok{1}\NormalTok{,}
                     \AttributeTok{na.strings =} \ConstantTok{NA}\NormalTok{)}

\FunctionTok{str}\NormalTok{(ppbiocsv)}
\NormalTok{ppbiocsv}
\end{Highlighting}
\end{Shaded}

Lembre de prestar atenção no dígito separador de decimais '' , '' ou ''
. '' . Além disso, só estaamos usando \texttt{ppbio**.***} porque o
diretório de trabalho ja fo definido no início. Se não deveríamos estar
usando \texttt{C:/Seu/Diretório/De/Trabalho/ppbio**.***}

Alguns comandos para exibir a planilha são ``case-sensitive''
\texttt{(ignore.case(object))}

\begin{Shaded}
\begin{Highlighting}[numbers=left,,]
\CommentTok{\#View(ppbio)}
\FunctionTok{print}\NormalTok{(ppbio)}
\NormalTok{ppbio}
\FunctionTok{str}\NormalTok{(ppbio)}
\CommentTok{\#?View}
\CommentTok{\#?view}
\CommentTok{\#?remove}
\end{Highlighting}
\end{Shaded}

\hypertarget{outra-forma-de-achar-e-importar-uma-planilha}{%
\subsection{Outra forma de achar e importar uma
planilha}\label{outra-forma-de-achar-e-importar-uma-planilha}}

Essa forma é desaconselhavel porque é demorada e sujeita a erros. Além
de precisar ser refeita sempre que se quiser abrir uma nova planilha ou
reabrir a última planilha importada. Veja o tópico
\protect\hyperlink{prefira-sempre-cuxf3digos-e-scripts-do-que-mouse-e-menus-de-janelas-no-r}{Prefira
sempre códigos e scripts do que mouse e menus de janelas no R}

\begin{Shaded}
\begin{Highlighting}[numbers=left,,]
\FunctionTok{getwd}\NormalTok{()}
\NormalTok{ppbio }\OtherTok{\textless{}{-}} \FunctionTok{read.xlsx}\NormalTok{(}\FunctionTok{file.choose}\NormalTok{(),  }\CommentTok{\#abre o windows explorer}
                   \AttributeTok{rowNames =}\NormalTok{ T, }\AttributeTok{colNames =}\NormalTok{ T,}
                   \AttributeTok{sheet =} \StringTok{"Sheet1"}\NormalTok{)}
\end{Highlighting}
\end{Shaded}

\hypertarget{importando-.ods-do-libreoffice-calc}{%
\section{\texorpdfstring{Importando \texttt{.ods} do LibreOffice
Calc}{Importando .ods do LibreOffice Calc}}\label{importando-.ods-do-libreoffice-calc}}

A planilha a seguir pode ser baixada da seção @ref(download401)

\begin{Shaded}
\begin{Highlighting}[numbers=left,,]
\FunctionTok{install.packages}\NormalTok{(}\StringTok{"readODS"}\NormalTok{)}
\end{Highlighting}
\end{Shaded}

\begin{Shaded}
\begin{Highlighting}[numbers=left,,]
\FunctionTok{library}\NormalTok{(readODS)}
\end{Highlighting}
\end{Shaded}

\begin{verbatim}
Warning: package 'readODS' was built under R version 4.3.3
\end{verbatim}

\begin{Shaded}
\begin{Highlighting}[numbers=left,,]
\NormalTok{ppbio06.ods }\OtherTok{\textless{}{-}} \FunctionTok{read\_ods}\NormalTok{(}\StringTok{"D:/Elvio/OneDrive/Disciplinas/\_EcoNumerica/5.Matrizes/ppbio06{-}peixes.ods"}\NormalTok{,}
                   \AttributeTok{row\_names =} \ConstantTok{TRUE}\NormalTok{,}
                   \AttributeTok{col\_names =} \ConstantTok{TRUE}\NormalTok{,}
                   \AttributeTok{sheet =} \StringTok{"Sheet1"}\NormalTok{,}
                   \AttributeTok{as\_tibble =} \ConstantTok{FALSE}\NormalTok{,}
                   \AttributeTok{na =} \StringTok{"n/a"}\NormalTok{, }\CommentTok{\# quando existem celulas vazias (n/a)}
\NormalTok{                   )}
\NormalTok{ppbio06.ods }\OtherTok{\textless{}{-}} \FunctionTok{na.omit}\NormalTok{(ppbio06.ods)}
\FunctionTok{str}\NormalTok{(ppbio06.ods)}
\FunctionTok{class}\NormalTok{(ppbio06.ods)}
\NormalTok{ppbio06.ods\_ma }\OtherTok{\textless{}{-}} \FunctionTok{as.matrix}\NormalTok{(ppbio06.ods) }\CommentTok{\#lê como uma matriz}
\FunctionTok{str}\NormalTok{(ppbio06.ods\_ma)}
\FunctionTok{class}\NormalTok{(ppbio06.ods\_ma)}
\CommentTok{\#ppbio06.ods}
\CommentTok{\#ppbio06.ods\_ma}
\end{Highlighting}
\end{Shaded}

\hypertarget{manipulando-tabelas-de-dados}{%
\section{Manipulando tabelas de
dados}\label{manipulando-tabelas-de-dados}}

\hypertarget{colunas-com-o-mesmo-nome}{%
\subsection{Colunas com o mesmo nome}\label{colunas-com-o-mesmo-nome}}

Em algumas situações é necessário encontrar colunas com o mesmo nome em
um conjunto de dados e soma-las ou fazer sua média. Isso pode ocorrer
quando se está trabalhando com dados de várias fontes e é necessário
combinar esses diferentes conjuntos de dados.

Considere por exemplo um cenário onde se está trabalhando em um projeto
de análise de dados ecológicos e você recebeu conjuntos de dados de
diferentes locais de amostragens enviados por diferentes pesquisadores,
e cada pesquiador enviou seu conjunto de dados que contém informações
sobre as mesmas espécies (ou variáveis ambientais). Devido a diferentes
sistemas de registro ou a falta de comunicação entre os pesquisadores,
pode haver repetição nos nomes das colunas.

\begin{Shaded}
\begin{Highlighting}[numbers=left,,]
\CommentTok{\#Dados de mamíferos roedores}
\NormalTok{df }\OtherTok{\textless{}{-}} \FunctionTok{data.frame}\NormalTok{(}
  \AttributeTok{Rato =} \FunctionTok{c}\NormalTok{(}\DecValTok{1}\NormalTok{, }\DecValTok{2}\NormalTok{, }\DecValTok{3}\NormalTok{),}
  \AttributeTok{Musaranho =} \FunctionTok{c}\NormalTok{(}\DecValTok{4}\NormalTok{, }\DecValTok{5}\NormalTok{, }\DecValTok{6}\NormalTok{),}
  \AttributeTok{Rato =} \FunctionTok{c}\NormalTok{(}\DecValTok{7}\NormalTok{, }\DecValTok{8}\NormalTok{, }\DecValTok{9}\NormalTok{), }\CommentTok{\#nome duplicado}
  \AttributeTok{Esquilo =} \FunctionTok{c}\NormalTok{(}\DecValTok{0}\NormalTok{, }\DecValTok{0}\NormalTok{, }\DecValTok{1}\NormalTok{),}
\NormalTok{  Ratão }\OtherTok{=} \FunctionTok{c}\NormalTok{(}\DecValTok{1}\NormalTok{, }\DecValTok{1}\NormalTok{, }\DecValTok{0}\NormalTok{),}
  \AttributeTok{Castor =} \FunctionTok{c}\NormalTok{(}\DecValTok{1}\NormalTok{, }\DecValTok{0}\NormalTok{, }\DecValTok{0}\NormalTok{),}
\NormalTok{  Tâmia }\OtherTok{=} \FunctionTok{c}\NormalTok{(}\DecValTok{11}\NormalTok{, }\DecValTok{12}\NormalTok{, }\DecValTok{13}\NormalTok{),}
  \AttributeTok{Marmota =} \FunctionTok{c}\NormalTok{(}\DecValTok{1}\NormalTok{, }\DecValTok{2}\NormalTok{, }\DecValTok{0}\NormalTok{),}
  \AttributeTok{Castor =} \FunctionTok{c}\NormalTok{(}\DecValTok{2}\NormalTok{, }\DecValTok{1}\NormalTok{, }\DecValTok{1}\NormalTok{), }\CommentTok{\#nome duplicado}
  \AttributeTok{check.names =} \ConstantTok{FALSE}
\NormalTok{)}
\NormalTok{df}
\end{Highlighting}
\end{Shaded}

\begin{verbatim}
  Rato Musaranho Rato Esquilo Ratão Castor Tâmia Marmota Castor
1    1         4    7       0     1      1    11       1      2
2    2         5    8       0     1      0    12       2      1
3    3         6    9       1     0      0    13       0      1
\end{verbatim}

Por exemplo, no estudo sobre mamíferos roedores acima, quando se combina
o conjunto geral de dados para realizar uma análise abrangente,
depara-se com colunas duplicadas, onde a matriz com o conjunto total de
dados contém espécies repetidas.

Nesse caso, encontrar e resolver colunas com o mesmo nome é crucial para
garantir a integridade dos dados e realizar uma análise precisa. Você
deve consolidar essas colunas duplicadas, somando-as ou fazendo sua
média.

\begin{Shaded}
\begin{Highlighting}[numbers=left,,]
\CommentTok{\# Achando colunas com nomes duplicados}
\NormalTok{dup\_cols }\OtherTok{\textless{}{-}} \FunctionTok{names}\NormalTok{(df)[}\FunctionTok{duplicated}\NormalTok{(}\FunctionTok{names}\NormalTok{(df))]}
\CommentTok{\# Somando colunas com o mesmo nome}
\ControlFlowTok{for}\NormalTok{ (col\_name }\ControlFlowTok{in} \FunctionTok{unique}\NormalTok{(dup\_cols)) \{}
  \CommentTok{\# Get indices of columns with the same name}
\NormalTok{  col\_indices }\OtherTok{\textless{}{-}} \FunctionTok{which}\NormalTok{(}\FunctionTok{names}\NormalTok{(df) }\SpecialCharTok{==}\NormalTok{ col\_name)}
  \CommentTok{\# Sum columns with the same name}
\NormalTok{  df[[col\_name]] }\OtherTok{\textless{}{-}} \FunctionTok{rowSums}\NormalTok{(df[, col\_indices, }\AttributeTok{drop =} \ConstantTok{FALSE}\NormalTok{])}
\NormalTok{\}}
\CommentTok{\# Remove as colunas duplicadas originais e mantem as novas colunas que são a soma ("except for the first occurrence")}
\NormalTok{df }\OtherTok{\textless{}{-}}\NormalTok{ df[, }\SpecialCharTok{!}\FunctionTok{duplicated}\NormalTok{(}\FunctionTok{names}\NormalTok{(df))]}
\CommentTok{\# Mostra a nova tabela com colunas repetidas somadas}
\FunctionTok{print}\NormalTok{(df)}
\end{Highlighting}
\end{Shaded}

\begin{verbatim}
  Rato Musaranho Esquilo Ratão Castor Tâmia Marmota
1    8         4       0     1      3    11       1
2   10         5       0     1      1    12       2
3   12         6       1     0      1    13       0
\end{verbatim}

\hypertarget{removendo-linhas-ou-colunas-por-nome}{%
\subsection{Removendo linhas ou colunas por
nome}\label{removendo-linhas-ou-colunas-por-nome}}

\begin{Shaded}
\begin{Highlighting}[numbers=left,,]
\CommentTok{\#Colunas}
\NormalTok{df }\OtherTok{\textless{}{-}} \FunctionTok{subset}\NormalTok{(ppbio06.ods, }\AttributeTok{select =} \SpecialCharTok{{-}}\NormalTok{la.chalumnae) }\CommentTok{\#escolhendo uma coluna pelo nome}
\CommentTok{\#Linhas}
\NormalTok{del\_rows }\OtherTok{\textless{}{-}} \FunctionTok{c}\NormalTok{(}\StringTok{"S{-}A{-}ZA1"}\NormalTok{, }\StringTok{"S{-}R{-}CC1"}\NormalTok{, }\StringTok{"B{-}R{-}ET1"}\NormalTok{)}
\NormalTok{del\_rows}
\NormalTok{m\_part }\OtherTok{\textless{}{-}}\NormalTok{ df[}\SpecialCharTok{!}\NormalTok{(}\FunctionTok{row.names}\NormalTok{(df) }\SpecialCharTok{\%in\%} \FunctionTok{c}\NormalTok{(del\_rows)),]}
\NormalTok{m\_part}
\end{Highlighting}
\end{Shaded}

\hypertarget{criando-uma-matriz-de-muxe9dias}{%
\subsection{Criando uma matriz de
médias}\label{criando-uma-matriz-de-muxe9dias}}

Por razões diferentes o precedimento anterior pode vir a ser necessário
de ser aplicado às linhas. Por exemplo, quando se quer somar ou fazer a
média de amostras diferentes do mesmo ambiente de coleta. Veja a matriz
abaixo.

\begin{Shaded}
\begin{Highlighting}[numbers=left,,]
\NormalTok{data }\OtherTok{\textless{}{-}} \FunctionTok{read.table}\NormalTok{(}\AttributeTok{text =} \StringTok{"}
\StringTok{Sp1 Sp2 Sp3 Sp4 Sp5 Sp6 Sp7 Sp8}
\StringTok{A1 0 0 0 0 0 0 6 1}
\StringTok{A2 0 0 0 2 0 0 10 2}
\StringTok{B1 93 2 0 177 0 260 2 5}
\StringTok{B2 0 4 0 8 0 0 83 7}
\StringTok{C1 0 0 0 0 1 0 0 1}
\StringTok{C2 0 0 1 0 0 1 0 1}
\StringTok{C3 0 2 0 2 0 0 0 1}
\StringTok{"}\NormalTok{, }\AttributeTok{header =} \ConstantTok{TRUE}\NormalTok{, }\AttributeTok{row.names =} \DecValTok{1}\NormalTok{)}
\NormalTok{data}
\end{Highlighting}
\end{Shaded}

\begin{verbatim}
   Sp1 Sp2 Sp3 Sp4 Sp5 Sp6 Sp7 Sp8
A1   0   0   0   0   0   0   6   1
A2   0   0   0   2   0   0  10   2
B1  93   2   0 177   0 260   2   5
B2   0   4   0   8   0   0  83   7
C1   0   0   0   0   1   0   0   1
C2   0   0   1   0   0   1   0   1
C3   0   2   0   2   0   0   0   1
\end{verbatim}

\begin{Shaded}
\begin{Highlighting}[numbers=left,,]
\FunctionTok{library}\NormalTok{(}\StringTok{"tidyverse"}\NormalTok{)}
\CommentTok{\#Inserindo coluna para agrupamentos}
\FunctionTok{nrow}\NormalTok{(data); }\FunctionTok{ncol}\NormalTok{(data) }\CommentTok{\#no. de N colunas x M linhas}
\NormalTok{data\_g }\OtherTok{\textless{}{-}} \FunctionTok{cbind}\NormalTok{(}\AttributeTok{Grupos =} \FunctionTok{rownames}\NormalTok{(data), data)}
\NormalTok{data\_g}

\NormalTok{grps }\OtherTok{\textless{}{-}} \FunctionTok{substr}\NormalTok{(data\_g[, }\DecValTok{1}\NormalTok{], }\DecValTok{1}\NormalTok{,}\DecValTok{3}\NormalTok{)}
\NormalTok{grps}

\NormalTok{data\_g }\OtherTok{\textless{}{-}}\NormalTok{ data\_g }\SpecialCharTok{\%\textgreater{}\%} \FunctionTok{mutate}\NormalTok{(}\AttributeTok{Grupos=}\FunctionTok{c}\NormalTok{(grps))}

\CommentTok{\#data\_avg \textless{}{-} aggregate(data\_g[, 9:9], list(data\_g$Grupos), mean)}
\CommentTok{\#data\_avg}

\NormalTok{data\_avg }\OtherTok{\textless{}{-}}\NormalTok{ data\_g }\SpecialCharTok{\%\textgreater{}\%} 
  \FunctionTok{group\_by}\NormalTok{(Grupos) }\SpecialCharTok{\%\textgreater{}\%}
  \FunctionTok{summarise}\NormalTok{(}\FunctionTok{across}\NormalTok{(}\AttributeTok{.cols =} \FunctionTok{everything}\NormalTok{(), }\SpecialCharTok{\textasciitilde{}} \FunctionTok{mean}\NormalTok{(.x, }\AttributeTok{na.rm =} \ConstantTok{TRUE}\NormalTok{)))}
\NormalTok{data\_avg}

\NormalTok{data\_dp }\OtherTok{\textless{}{-}}\NormalTok{ data\_g }\SpecialCharTok{\%\textgreater{}\%} 
  \FunctionTok{group\_by}\NormalTok{(Grupos) }\SpecialCharTok{\%\textgreater{}\%}
  \FunctionTok{summarise}\NormalTok{(}\FunctionTok{across}\NormalTok{(}\AttributeTok{.cols =} \FunctionTok{everything}\NormalTok{(), }\FunctionTok{list}\NormalTok{(}\AttributeTok{mean =}\NormalTok{ mean, }\AttributeTok{sd =}\NormalTok{ sd)))}
\CommentTok{\#?across}
\NormalTok{data\_dp}

\CommentTok{\#Primeira coluna para nomes das linhas }
\NormalTok{data\_dp }\OtherTok{\textless{}{-}} \FunctionTok{as.data.frame}\NormalTok{(data\_dp)}
\FunctionTok{class}\NormalTok{(data\_dp)}
\FunctionTok{rownames}\NormalTok{(data\_dp) }\OtherTok{\textless{}{-}}\NormalTok{ data\_dp[,}\DecValTok{1}\NormalTok{]}
\NormalTok{data\_dp[,}\DecValTok{1}\NormalTok{] }\OtherTok{\textless{}{-}} \ConstantTok{NULL}
\NormalTok{data\_dp }\OtherTok{\textless{}{-}} \FunctionTok{round}\NormalTok{(data\_dp, }\DecValTok{1}\NormalTok{)}
\NormalTok{data\_dp}
\CommentTok{\#Salvando a matriz}
\FunctionTok{write.table}\NormalTok{(data\_dp,}
            \StringTok{"data\_dp.csv"}\NormalTok{,}
            \AttributeTok{append =}\NormalTok{ F,}
            \AttributeTok{quote =} \ConstantTok{TRUE}\NormalTok{,}
            \AttributeTok{sep =} \StringTok{";"}\NormalTok{, }\AttributeTok{dec =} \StringTok{","}\NormalTok{,}
            \AttributeTok{row.names =}\NormalTok{ T)}
\NormalTok{data\_dp\_csv }\OtherTok{\textless{}{-}} \FunctionTok{read.csv}\NormalTok{(}\StringTok{"data\_dp.csv"}\NormalTok{,}
                    \AttributeTok{sep =} \StringTok{";"}\NormalTok{, }\AttributeTok{dec =} \StringTok{","}\NormalTok{,}
                    \AttributeTok{header =}\NormalTok{ T,}
                    \AttributeTok{row.names =} \DecValTok{1}\NormalTok{,}
                    \AttributeTok{na.strings =} \ConstantTok{NA}\NormalTok{)}
\end{Highlighting}
\end{Shaded}

\hypertarget{referuxeancias}{%
\section*{Referências}\label{referuxeancias}}
\addcontentsline{toc}{section}{Referências}

\hypertarget{refs}{}
\begin{CSLReferences}{1}{0}
\leavevmode\vadjust pre{\hypertarget{ref-RN2774}{}}%
R Core Team. 2017. \href{https://www.r-project.org/}{R: A language and
environment for statistical computing}. Book, R Foundation for
Statistical Computing, Austria.

\leavevmode\vadjust pre{\hypertarget{ref-RN358}{}}%
Team, Rs. 2022. \href{http://www.rstudio.com/}{RStudio: Integrated
Development Environment for R}. Book, RStudio, PBC, Boston, MA.

\end{CSLReferences}

\hypertarget{apuxeandices}{%
\section*{Apêndices}\label{apuxeandices}}
\addcontentsline{toc}{section}{Apêndices}

\hypertarget{sites-para-consulta}{%
\subsection*{Sites para consulta}\label{sites-para-consulta}}
\addcontentsline{toc}{subsection}{Sites para consulta}

Como importar dados do Excel para o R:
\url{https://youtu.be/U6ksXvvY6Q0}\\
Como exportar dados do R para o Excel:
\url{https://youtu.be/a7EJE_2mtGk}

\hypertarget{script-limpo}{%
\section*{Script limpo}\label{script-limpo}}
\addcontentsline{toc}{section}{Script limpo}

Aqui apresento o scrip na íntegra sem os textos ou outros comentários.
Você pode copiar e colar no R para executa-lo. Lembre de remover os
\texttt{\#} ou \texttt{\#\#} caso necessite executar essas linhas.



\end{document}
